% !TeX root = ../main.tex
% Add the above to each chapter to make compiling the PDF easier in some editors.

\chapter{Related Works}\label{chapter:relatedworks}

The ability to perform spatial reasoning in a scene is a critical step for performing complex functions. Estimating human pose from a 2D image can be thought as problem related to detecting the position of the joints in the image plane and predicting the depth of each joint. Key point detection and depth estimation are two important parts of spatial reasoning and are well established fields of Computer Vision. 

Geometrical reasoning on how 3D to 2D projections are related have been explored with the lens of estimating lengths and distance ratios between objects in the scene \parencite{criminisi2000single} and estimating the geometric structure of the scene using line segments \parencite{lee2009geometric}. These show that regular structures like line segments and prior information on parallel structures contain a lot of valuable information about depth and perspective. Depth estimation from monocular cues like texture \parencite{lindeberg1993shape}, shading \parencite{zhang1999shape} and learning based approaches which combine them \parencite{saxena2006learning} are also important areas of research. More recently Supervised Deep Learning based methods have been using monocular cues which outperform all previous methods \parencite{eigen2014depth}, \parencite{eigen2015predicting}, \parencite{liu2015deep}, \parencite{liu2016learning}. These methods illustrate that visual information in the form of monocular cues also contain valuable information which can be used for depth estimation. There is also another line of research which utilizes stereo cameras with geometric consistency priors in an Unsupervised Deep Learning framework which are giving very promising results \parencite{garg2016unsupervised}, \parencite{godard2017unsupervised}. These show that depth can be perceived from matching of key points and disparity calculations. 

Although, human vision uses a combination of lower level processes like depth perception and higher level processes like object recognition, higher level processes play a much more important role when building mental object representations and models of hierarchical organization of visual processing \parencite{bulthoff1998top}. Familiar 3D structures can be matched to 2D projections even when there are conflicting depth cues. This gives motivation to pursue a line of study in human 3D pose estimation which only rely on joint positions and structural priors. 

The related works section is divided into three parts. First part discusses the different pose representation methods. Second part briefly discusses various approaches to 2D human pose estimations. The last part examines different techniques for 3D human pose estimation 

\section{Pose Representation}

There is a variety of spatio-temporal representation methods for 3D human poses. Humans visual perception of 3D human poses is not better than state-of-the-art Computer Vision algorithms when calculated in conventional metrics. Humans recognize and re-enact 3D human poses with 10-20 degree or 100 mm per joint error \parencite{marinoiu2013pictorial}. This shows that the way human pose is represented and the way the error metrics are calculated are still open problems.

\subsection{Kinematic Tree}

Approaches which use the kinematic tree model \parencite{barron2001estimating}, \parencite{wei2009modeling}, \parencite{zhou2016deep}, \parencite{sun2017compositional}, \parencite{mehta2017monocular} represent the pose in terms of a root joint and hyarerchical pairwise relations that use the length of the limb and joint angles to point from the parent joint to its child. This model has the advantage that it can impose  skeletal structural limits constraints like anthropometric proportions, joint angle limits, symmetry conditions and rigid body limits \parencite{dabral2017structure}, \parencite{wei2009modeling}. This can constrain the total degrees of freedom in the model and anatomically sound pose estimates. For an example please look at Figure 2.1. SMPL \parencite{loper2015smpl} is statistical human body model which represents pose in a kinematic tree tailored to the subject’s body shape. It combines information about pose and body shape in a single model and it has shown promise in 3D pose estimation \parencite{bogo2016keep}.

\subsection{Pose Dictionary}

This approach represents the pose as a combination of known poses. The dictionary of known basis poses can be gather statistically from a 3D pose dataset using techniques like PCA. The matching can be done using an optimization framework  \parencite{ramakrishna2012reconstructing}, \parencite{zhou20153d} or more recently with the help of a Deep Learning model \parencite{zhou2016sparseness}, \parencite{chen20173d}, \parencite{tome2017lifting}. This methodology has the drawback that each pose can be represented in multiple ways and can lead to invalid 3D poses. This can be mitigated by jointly optimizing a pose matching and joint angle limit constraints \parencite{akhter2015pose}.

\subsection{Model-free}

Regressing the joint positions directly has become more popular in the recent work which use Deep Learning models , \parencite{pavlakos2017coarse}, \parencite{martinez2017simple}, \parencite{hossain2017exploiting}, \parencite{tekin2017learning}. These methods are much more simpler in the pose representation and rely on the Deep Learning models to learn the structural information in the human body. One downside of this methods is the lack of structural priors can lead the model to make anthropometrically invalid 3D pose estimations.

\section{2D Pose Estimation}

Since our solution pipeline starts with estimating 2D poses from monocular images we will discuss the prominent techniques in this area. 2D pose estimation aims to localize certain number of joints in the image. This information although ambiguous on its own can be used in later stages to estimate the full 3D pose of the person.

One of the most prominent results in 2D pose estimation have been obtained by Newell et. at \parencite{newell2016stacked}. He uses repeated top-down and bottom-up processing steps in conjunction with intermediate supervision to achieve very strong results. Each block in the network is called hourglass and consists of encoding and decoding layers. Each hourglasses are stacked on top of each other and refine each other's output. The whole architecture is trained end-to-end.

Many methods build on top of the Stacked Hourglass model. \parencite{chu2017multi} use Conditional Random Fields to associate neighboring regions and body part attention model to impose global consistency of the body. \parencite{chou2017self} uses adversarial training to give a meaningful structural priors for the body poses. The discriminator helps the model reject invalid joints angles and anthropometrically invalid poses.

\parencite{iqbal2017posetrack} and \parencite{insafutdinov2017arttrack} attempted estimating the pose for multiple people in  videos. Both approaches are closely related but differ in the type of body-part proposals and the structure of the spatio-temporal graph where Iqbal et. al. relies on \parencite{insafutdinov2016deepercut} for body-part detectors Insafutdinov et. al. use person-conditioned model that is trained to associate body parts of a specific person already at the detection stage. 

\parencite{cao2016realtime} utilized a CNN based model similar to the Stacked Hourglass model where they refine their predictions in multiple steps. They added another branch to the model which predicts Part Affinity Fields (PAFs). This enables multi-person pose estimation by predicting association fields between joints. PAFs are later used to match the joints in a bottom-up process. This algorithm can run in real-time on commodity hardware.

\section{3D Pose Estimation}

There are many different approaches to 3D human pose estimations. Recently some methods use Deep Neural Networks to estimate the 3D pose directly from images \parencite{pavlakos2017coarse}, \parencite{tekin2016structured}, \parencite{varol2017learning}, \parencite{rogez2016mocap}. Some models have tried to formulate the problems as a retrieval task. Similar poses are search from a dictionary and then combined or refined to get the final pose []. Another line of research uses either ground truth 2D poses or estimates from a 2D pose estimator to train a model that lifts the 2D poses to 3D. Other methods use additional information besides the given image like temporal information [] or structural priors []. We will discuss these methods in more detail below.

\subsection{Deep network trained end-to-end}

The emergence of the Human3.6M \parencite{ionescu2014human3} enabled the use of end-to-end trained Deep Learning models to estimate 3D human poses direcly from images. Although these methods rely on a laboratory setup which doesn't generalize well to the in-the-wild scenes algorithmic innovation has shown many advances. The Human3.6M database benchmark has been constantly imporoving eversince.

Pavlakos et al. \parencite{pavlakos2017coarse} predicts volumetric heatmaps for the joints and uses an interative refinement strategy to imporove the heatmaps. Tekin et. al. \parencite{tekin2016structured} tries to the problem of anthropometrically invalid poses by training a de-noising auto-encoder encode the human pose in an embedding. The auto-encoder is stacked on top of a Convolutional Neural Network and later fine tuned together end-to-end.

Another line of research focuses of synthesizing novel frames by augmentation. Varol et. al. \parencite{varol2017learning}showed sthat learning from synthetic scenes can be effective in especcially when combined with real scenes to predict depth and body segmentation. \parencite{rogez2016mocap} generated synthetic images by blending real images  together and showed that this techniques improves prediction accuracy.

Following sections show methods which improve the direct estimation methods by utilizing cobined training of 2D and 3D poses, temporal information or structural priors.

\subsection{Pose Dictionaries}

Some approaches utilize a 3D pose dictionary to look up similar poses. This can take the form of an embedding or collection of joint positions which are then refined using an optimization mechanism.

Rogez et. al. \parencite{rogez2017lcr} build an anchor-pose dictionary by clustering a subset of the training images. A region porposal network \parencite{ren2015faster} network outputs anchor-pose proposals given an image. Another network scores the anchor-poses in one branch by a classification loss and refines them in another branch using a regession loss. Finally the refined and scored anchor poses are integrated to give the 3D pose estimate.

Zhou et. al. \parencite{zhou2016sparseness} use a Convolutional Neural Network to estimate a sequence of joint heatmaps and use the Expectation-Maximization algorithm to estimate the 3D pose out of a 3D pose dictionary. 

In \parencite{chen20173d} Chen et. al. assume the conditional independance of 3D pose from the image given the 2D pose. This makes their solution a 2 part solution. They estimate the 2D pose with a Convolutional Neural Network and use a Non-parametric Nearest Neighbor model to match the 2D pose to a known 3D pose. 

The accuracy of these models depend on the size of the Pose Dictionary. No matter how large the pose dictionary there will be still unobserved poses on the tail end of the pose distribution. This makes making inference and finding a large enough dictionary prohibitively expensive.

\subsection{2D Information}

2D Pose estimation has seen incredible progress in recent years \parencite{cao2016realtime}, \parencite{newell2016stacked}, \parencite{iqbal2017posetrack}. 2D pose estimation is a necesary part of 3D pose estimation which lead many reasearches to either use 2D pose estimation as an additional supervision signal or decouple the two problems and use the 2D poses as input to the 3D pose estimation problem. Both methods have shown remarkable improvements over the baseline since they can eddectively utilize the abundant 2D pose datasets. 

Park et. al. \parencite{park20163d} used a Deep Learning model which estimatmates both 2D and 3D pose. While the 3D pose estimation is formulated as a regression problem the 2D pose estimation is formulated as a classification problem where the image is tiled into grids and each joint is classified to one grid. The search space for 3D pose is much larger which prevents the use of the grid tiling. The network has a Convolutional Neural Network architecture where it branches out in the fully connected layers into two parts. The 3D joint locations are predicted by combining multiple candidate poses which act as an ensemble mechanism.

Tekin et. al. \parencite{tekin2017learning} use a Convolutional Neural Network to estimate 2D confidance maps for each jont. Those confidence maps and the original image are fed into 2 branches of a Convolutional Neural Network and their feature maps are fused together to estimate the final 3D pose. A similar machanism is performed by Tome et. al. \parencite{tome2017lifting}. Their approaches differ mainly in that Tome et. al. doesn't seperate 2D pose estimation to a seperate branch. They use a iterative refining mechanism similar how \parencite{newell2016stacked} used where in each stage they predict 2D confidence maps for the joints, estimate 3D pose from those, project the 3D pose into 2D and fuse the projected 2D pose estimate with the 2D confidence map and apply a reprojection error. After the iterative refinement the lifting operation which gives the final 3D pose. Nie et. al. \parencite{nie2017monocular} train 2 networks seperately a Skeleton Long-Short Term Memory Network \parencite{hochreiter1997long} which estimates global 3D pose given 2D pose and a Patch Long-Short Term Memory Network which given image patches around the joints estimates the depth of those joints. The two network are later integrated in the second layer Long-Short Term Memory Network by combining the output of the two and estimating the depth of the joints. The architecture has a multi-task with depth prediction loss and 2D pose estimation loss which enables the use of in-the-wild 2D pose databases. 

Moreno et. al. \parencite{moreno20173d} uses the output of a off-the-shelf 2D pose detector as input. This has the advantage of profoundly simplifying the estimation process. First they compute the Euclidian Distance Matrix (EDM) which encodes the distances between each joint. The EDM is fed into a Convolutional Neural Network which estimates a Euclidian Distance Matrix for 3D pose. This is used to predict the 3D joint locations using Multidimentional Scaling. Martinez et. al. \parencite{martinez2017simple} follow a similar approach where they use the output of a off-the-shelf 2D pose detector. The train a simple 4 layer feed-forward neural network with residual connections to lift the 2D pose into 3D. This simplifies the training procedure immensely without loosing the accuracy benefits and enables them to accurately estimate the pose in images which are visually demanding.

\subsection{Temporal Information}

\parencite{hossain2017exploiting}
\parencite{hossain2017understanding}

Lin et. al. \parencite{lin2017recurrent} utilizes an iterative refinement mechanism using Long-Short Term Memory Networks (LSTM) \parencite{hochreiter1997long}. This enables them to both utilize temporal information and use the critical refinement strategy which many of the state-of-the-art methods rely on heavily. They designed a 2D pose module which estimates image dependant pose representations the feature maps are passed to both a feature adaptation module which lift the feature maps to 3d and to the next stage through convolutional filters. 

Lin et al. [65] predicts 3D pose from an image directly and refines them in multiple
stages using LSTM [48]. Each stage has a 2D pose module which learns a two di-
mensional pose-aware feature map that encodes information of human body pose.
This feature map is passed onto feature adaptation module which gives a high di-
mensional common embedding space for 2D and 3D pose. The adapted feature is
concatenated with the hidden states of the LSTM and 3D pose detection from the
previous stage and is passed as input to the LSTM of current stage to predict the
3D pose in the current refinement stage.

Estimating 3D pose per frame may cause jitter because the error in pose estimation
for each frame is independent of one another. A natural extension would be to
estimate the 3D pose over a sequence of images or monocular video such that the
poses look temporally coherent and smooth i.e. the error is distributed smoothly
over a sequence. A number of methods tried to exploit the temporal information
available over a sequence of images to achieve temporal smoothness.
Andriluka et al. [4] exploited temporal information using tracking-by-detection.
They first estimated 2D poses for each frame individually. Then they associated the
poses across frames using tracking-by-detection method. The robust estimates of
2D pose over a short sequence was used to recover 3D pose. Tekin et al. [117] ex-
ploited the motion information by first using a CNN to align successive bounding
boxes such that the person always remains in the center of the bounding box. Then
they concatenated the aligned images and extracted 3D HOG (histogram of gra-
dients) features densely over the spatio-temporal volume from which they regress
the 3D pose of the central frame. They tried different techniques for regressing 3D
pose and found deep network to work the best. Du et al. [29] used a height-map,
estimated from RGB image and camera calibration, and RGB image to regress 2D
joint locations using dual stream CNN. From a sequence of 2D joints, they esti-
mated 3D pose by minimizing reprojection error and by imposing pose-conditioned
joint velocity and temporal coherence constraints during optimization. Mehta et al.
[74] implemented a real time system for 3D pose estimation which exploits tem-
poral information from the previous frame to achieve temporal smoothness. Given
an image the bounding box at time t is estimated by tracking the bounding box
and 2D joint locations of the previous frame which is passed to a CNN to estimate
2D heatmaps and 3D location map x,y,z for each joint. They combine the 2D and
263D pose predictions of the current frame with that of the previous frame and apply
temporal filtering and smoothing to obtain the 3D pose of the current frame.
In our third model we exploit the temporal information present in a sequence
of frames and would like to examine if applying temporal constraints can improve
the performance of our previous network. For monocular videos, it is intuitive to
exploit the temporal information of previous frames as it can provide many impor-
tant cues like some part being occluded in one frame may be visible in the next
frame or in our case, the 2D pose estimation of a particular frame may be more er-
roneous than other frame. We expect that the temporal information will distribute
the error in pose estimation smoothly over the sequence reducing jitter and overall
improvement in results.

\subsection{Structrual Priors}

\parencite{bogo2016keep}
Bogo et al. [16] used the 2D joint heatmaps from a CNN-based 2D pose detec-
tor to predict both 3D pose and the 3D shape of human body. Their body model
is defined as a function parameterized by coefficients of shape prior, pose parame-
ters defined by kinematic tree model (See Section 2.1) and translation parameters.
They minimize five different error terms: joint-based error defined by re-projection
error under weak perspective projection, three pose priors and a shape prior. 
